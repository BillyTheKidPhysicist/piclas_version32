% =====================================================================================================
% PACKAGES and Settings
% =====================================================================================================

\usepackage{tikz}
\usepackage{xifthen}
\usetikzlibrary{spy,arrows}
\usetikzlibrary{arrows.meta}
\usetikzlibrary{shapes.geometric}
\usetikzlibrary{decorations.markings}
\usetikzlibrary{intersections}
\usetikzlibrary{positioning}
\usetikzlibrary{backgrounds}
\usepackage{pgfplots}
\pgfplotsset{compat=1.7}
\usetikzlibrary{calc}
\tikzstyle{rounded}  = [rounded corners]     
\tikzstyle{function} = [rectangle, text centered, draw=black,minimum width=6.0cm,minimum height=1.0cm]
\tikzstyle{mpi}      = [anchor=south west, xshift=0.2cm] 

% fuer alle Beispiele benutze N=3 => vier Gausspunkte ...
\newcommand{\xGP}{-0.866, -0.339,  0.339, 0.866}
\newcommand{\wGP}{0.348, 0.652, 0.652, 0.348}
\newcommand{\innerborders}{-0.652,0,0.652}
\newcommand{\xGPeq}{-0.75, -0.25,  0.25, 0.75}
\newcommand{\wGPeq}{0.5, 0.5, 0.5, 0.5}
\newcommand{\innerborderseq}{-0.5,0,0.5}

\usetikzlibrary{3d}

\usetikzlibrary{fillbetween}
\usetikzlibrary{decorations,decorations.text,backgrounds}

\usepackage{pgfplotstable}
\pgfplotsset{filter discard warning=false,compat=newest,/pgf/number format/.cd,1000 sep={}}         
\usepgfplotslibrary{colormaps,external}




\usetikzlibrary{calc}


\usepgfplotslibrary{patchplots} 
\usepgflibrary{plothandlers}


\usepackage{tikz-3dplot}       % 3D axis plotting
\tdplotsetmaincoords{60}{125}
\tdplotsetrotatedcoords{0}{0}{0} %<- rotate around (z,y,z)
\usetikzlibrary{plotmarks}

\usepackage{gnuplot-lua-tikz}
\usetikzlibrary{shapes,arrows,fadings,shadows}
\usetikzlibrary{decorations.pathreplacing,decorations.markings}





\usetikzlibrary{lindenmayersystems}



% =====================================================================================================
% Commands
% =====================================================================================================
\newcommand{\dittotikz}{%
    \tikz{
        \draw [line width=0.12ex] (-0.2ex,0) -- +(0,0.8ex)
            (0.2ex,0) -- +(0,0.8ex);
        \draw [line width=0.08ex] (-0.6ex,0.4ex) -- +(-1.5em,0)
            (0.6ex,0.4ex) -- +(1.5em,0);
    }%
  }






% =====================================================================================================
% MACROS
% =====================================================================================================




%-----------------------------------------------------------------------------------------------------
% for EOC triangle in log-log-plots
%-----------------------------------------------------------------------------------------------------

%------------------------------------------------------
%%% START MACRO FOR ANNOTATION OF TRIANGLE WITH SLOPE %%%.
\newcommand{\logLogSlopeTriangleBelow}[7]
{
    % #1. Relative offset in x direction.
    % #2. Width in x direction, so xA-xB.
    % #3. Relative offset in y direction.
    % #4. Slope d(y)/d(log10(x)).
    % #5. Plot options.

    \pgfplotsextra
    {
        \pgfkeysgetvalue{/pgfplots/xmin}{\xmin}
        \pgfkeysgetvalue{/pgfplots/xmax}{\xmax}
        \pgfkeysgetvalue{/pgfplots/ymin}{\ymin}
        \pgfkeysgetvalue{/pgfplots/ymax}{\ymax}

        % Calculate auxilliary quantities, in relative sense.
        \pgfmathsetmacro{\xArel}{#1}
        \pgfmathsetmacro{\yArel}{#3}
        \pgfmathsetmacro{\xBrel}{#1-#2}
        \pgfmathsetmacro{\yBrel}{\yArel}
        \pgfmathsetmacro{\xCrel}{\xArel}
        %\pgfmathsetmacro{\yCrel}{ln(\yC/exp(\ymin))/ln(exp(\ymax)/exp(\ymin))} % REPLACE THIS EXPRESSION WITH AN EXPRESSION INDEPENDENT OF \yC TO PREVENT THE 'DIMENSION TOO LARGE' ERROR.

        \pgfmathsetmacro{\lnxB}{\xmin*(1-(#1-#2))+\xmax*(#1-#2)} % in [xmin,xmax].
        \pgfmathsetmacro{\lnxA}{\xmin*(1-#1)+\xmax*#1} % in [xmin,xmax].
        \pgfmathsetmacro{\lnyA}{\ymin*(1-#3)+\ymax*#3} % in [ymin,ymax].
        \pgfmathsetmacro{\lnyC}{\lnyA+#4*(\lnxA-\lnxB)}
        \pgfmathsetmacro{\yCrel}{\lnyC-\ymin)/(\ymax-\ymin)} % THE IMPROVED EXPRESSION WITHOUT 'DIMENSION TOO LARGE' ERROR.

        % Define coordinates for \draw. MIND THE 'rel axis cs' as opposed to the 'axis cs'.
        \coordinate (A) at (rel axis cs:\xArel,\yArel);
        \coordinate (B) at (rel axis cs:\xBrel,\yBrel);
        \coordinate (C) at (rel axis cs:\xCrel,\yCrel);

        % Draw slope triangle.
        \draw[#5]   (A)-- node[pos=0.5,anchor=#6] {1}
                    (B)-- 
                    (C)-- node[pos=0.5,anchor=#7] {#4}
                    cycle;
    }
}
%%% END MACRO FOR ANNOTATION OF TRIANGLE WITH SLOPE %%%.



%------------------------------------------------------

%%% START MACRO FOR ANNOTATION OF TRIANGLE WITH SLOPE %%%.
\newcommand{\logLogSlopeTriangleAbove}[7]
{
    % #1. Relative offset in x direction.
    % #2. Width in x direction, so xA-xB.
    % #3. Relative offset in y direction.
    % #4. Slope d(y)/d(log10(x)).
    % #5. Plot options.

    \pgfplotsextra
    {
        \pgfkeysgetvalue{/pgfplots/xmin}{\xmin}
        \pgfkeysgetvalue{/pgfplots/xmax}{\xmax}
        \pgfkeysgetvalue{/pgfplots/ymin}{\ymin}
        \pgfkeysgetvalue{/pgfplots/ymax}{\ymax}

        % Calculate auxilliary quantities, in relative sense.
        \pgfmathsetmacro{\xArel}{#1}
        \pgfmathsetmacro{\yArel}{#3}
        \pgfmathsetmacro{\xBrel}{#1-#2}
        \pgfmathsetmacro{\yBrel}{\yArel}
        \pgfmathsetmacro{\xCrel}{\xArel}
        %\pgfmathsetmacro{\yCrel}{ln(\yC/exp(\ymin))/ln(exp(\ymax)/exp(\ymin))} % REPLACE THIS EXPRESSION WITH AN EXPRESSION INDEPENDENT OF \yC TO PREVENT THE 'DIMENSION TOO LARGE' ERROR.

        \pgfmathsetmacro{\lnxB}{\xmin*(1-(#1-#2))+\xmax*(#1-#2)} % in [xmin,xmax].
        \pgfmathsetmacro{\lnxA}{\xmin*(1-#1)+\xmax*#1} % in [xmin,xmax].
        \pgfmathsetmacro{\lnyA}{\ymin*(1-#3)+\ymax*#3} % in [ymin,ymax].
        \pgfmathsetmacro{\lnyC}{\lnyA+#4*(\lnxA-\lnxB)}
        \pgfmathsetmacro{\yCrel}{\lnyC-\ymin)/(\ymax-\ymin)} % THE IMPROVED EXPRESSION WITHOUT 'DIMENSION TOO LARGE' ERROR.

        % Define coordinates for \draw. MIND THE 'rel axis cs' as opposed to the 'axis cs'.
        \coordinate (A) at (rel axis cs:\xBrel,\yCrel);
        \coordinate (B) at (rel axis cs:\xBrel,\yBrel);
        \coordinate (C) at (rel axis cs:\xCrel,\yCrel);

        % Draw slope triangle.
        \draw[#5]   (A)-- node[pos=0.5,anchor=#6] {#4} %north
                    (B)-- 
                    (C)-- node[pos=0.5,anchor=#7] {1} % west
                    cycle;
    }
}
%%% END MACRO FOR ANNOTATION OF TRIANGLE WITH SLOPE %%%.


