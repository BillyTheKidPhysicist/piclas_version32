\renewcommand*\familydefault{\sfdefault}

\usepackage{datetime2}
\usepackage{fancyhdr}
\pagestyle{fancy}
\pagenumbering{arabic}
%\lhead{\@title}
\chead{}
\rhead{\itshape{\nouppercase{\leftmark}}}
\lfoot{R\Today}
\cfoot{}
\rfoot{\thepage}

\usepackage{lstautogobble}
\usepackage[dvipsnames]{xcolor} % for color list see https://en.wikibooks.org/wiki/LaTeX/Colors

\definecolor{very-light-gray}{gray}{0.9}
\definecolor{light-gray}{gray}{0.85}
\definecolor{medium-gray}{gray}{0.65}
%\definecolor{mycolor}{rgb}{0.38,0.1,0.82}

% Bash code written with spaces
\lstdefinestyle{defaultstyle}{ %
  backgroundcolor=\color{very-light-gray}, % choose the background color; you must add \usepackage{color} or \usepackage{xcolor}
  basicstyle=\ttfamily\footnotesize,   % the size of the fonts that are used for the code
  breakatwhitespace=false,           % sets if automatic breaks should only happen at whitespace
  breaklines=true,                   % sets automatic line breaking
  %captionpos=b,                      % sets the caption-position to bottom
  commentstyle=\color{ForestGreen},  % comment style
  deletekeywords={...},              % if you want to delete keywords from the given language
  escapeinside={\%*}{*)},            % if you want to add LaTeX within your code
  extendedchars=false,               % lets you use non-ASCII characters; for 8-bits encodings only, does not work with UTF-8
  frame=trbl,	                       % adds a frame around the code
  frameround=ffff,	                 % rounds frame corners t/f
  framexleftmargin=0pt,              % frame margins
  framexrightmargin=0pt,
  framextopmargin=0pt,
  framexbottommargin=0pt,
  keepspaces=true,                   % keeps spaces in text, useful for keeping indentation of code (possibly needs columns=flexible)
  keywordstyle=\ttfamily\small,      % keyword style
  language=bash,                     % the language of the code
  otherkeywords={*,...},             % if you want to add more keywords to the set
  rulecolor=\color{medium-gray},     % if not set, the frame-color may be changed on line-breaks within not-black text (e.g. comments (green here))
  showspaces=false,                  % show spaces everywhere adding particular underscores; it overrides 'showstringspaces'
  showstringspaces=false,            % underline spaces within strings only
  showtabs=false,                    % show tabs within strings adding particular underscores
  stringstyle=,                      % string literal style
  tabsize=2,	                       % sets default tabsize to 2 spaces
  %title=\lstname,                    % show the filename of files included with \lstinputlisting; also try caption instead of title
  aboveskip=0.7em,                   % margin above code block 
  belowskip=0.0em                   % margin below code block
}
\lstset{escapechar=@,style=defaultstyle}
\lstset{morecomment=[l]{!}}          % Fortran style comments

%% Code written in code environment  ~~~ <code> ~~~
%\lstdefinestyle{fort}{ %
%  backgroundcolor=\color{light-gray}, % choose the background color; you must add \usepackage{color} or \usepackage{xcolor}
%  basicstyle=\ttfamily\small,        % the size of the fonts that are used for the code
%  breakatwhitespace=false,           % sets if automatic breaks should only happen at whitespace
%  breaklines=true,                   % sets automatic line breaking
%  captionpos=b,                      % sets the caption-position to bottom
%  commentstyle=\color{blue},        % comment style
%  deletekeywords={...},              % if you want to delete keywords from the given language
%  escapeinside={\%*}{*)},            % if you want to add LaTeX within your code
%  extendedchars=false,               % lets you use non-ASCII characters; for 8-bits encodings only, does not work with UTF-8
%  frame=trbl,	                       % adds a frame around the code
%  frameround=ffff,	                 % rounds frame corners t/f
%  framexleftmargin=0pt,              % frame margins
%  framexrightmargin=0pt,
%  framextopmargin=0pt,
%  framexbottommargin=0pt,
%  keepspaces=true,                   % keeps spaces in text, useful for keeping indentation of code (possibly needs columns=flexible)
%  keywordstyle=\color{blue},         % keyword style
%  language=fortran,                  % the language of the code
%  otherkeywords={*,...},             % if you want to add more keywords to the set
%  numbers=left,                      % where to put the line-numbers; possible values are (none, left, right)
%  numbersep=5pt,                     % how far the line-numbers are from the code
%  numberstyle=\tiny,                 % the style that is used for the line-numbers
%  stepnumber=2,                      % the step between two line-numbers. If it's 1, each line will be numbered
%  rulecolor=\color{medium-gray},     % if not set, the frame-color may be changed on line-breaks within not-black text (e.g. comments (green here))
%  showspaces=false,                  % show spaces everywhere adding particular underscores; it overrides 'showstringspaces'
%  showstringspaces=false,            % underline spaces within strings only
%  showtabs=false,                    % show tabs within strings adding particular underscores
%  stringstyle=,                      % string literal style
%  tabsize=2,	                       % sets default tabsize to 2 spaces
%  title=\lstname                     % show the filename of files included with \lstinputlisting; also try caption instead of title

