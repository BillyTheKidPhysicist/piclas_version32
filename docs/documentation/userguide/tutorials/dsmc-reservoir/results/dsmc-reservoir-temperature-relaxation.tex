\documentclass[varwidth]{standalone}

\usepackage{tikz, pgfplots}
\usepackage[exponent-product = \cdot]{siunitx}

% Set default column separator for table (csv files)
\pgfplotsset{table/col sep=comma}
% Space as thousand separator.
\pgfkeys{/pgf/number format/.cd,1000 sep={\,}}

\begin{document}
\begin{tikzpicture}
\begin{axis}[
width=5.0cm,
height=5.0cm,
scale only axis,
scaled ticks=false,
unbounded coords=jump,
filter discard warning=false,
xmin=0,
xmax=5,
ymin=5000,
ymax=10000,
xlabel style={name=xlabel},
xlabel={$t$ $\left[\SI{e-6}{s}\right]$},
ylabel style={name=ylabel},
ylabel={Temperature $T$ $\left[\si{K}\right]$},
axis x line*=bottom,
axis y line*=left,
legend style={legend cell align=left, align=left},
%each nth point=65
]

\addplot[color=black]  table [col sep=comma, x=TIME, y=003-TempTra-001] {./dsmc-reservoir-temperature-relaxation.csv};
\addlegendentry{$T_\text{trans}$};
\addplot[dotted,color=black]  table [col sep=comma, x=TIME, y=004-TempVib001] {./dsmc-reservoir-temperature-relaxation.csv};
\addlegendentry{$T_\text{vib}$};
\addplot[dashed,color=black]  table [col sep=comma, x=TIME, y=006-TempRot001] {./dsmc-reservoir-temperature-relaxation.csv};
\addlegendentry{$T_\text{rot}$};

\end{axis}
\draw [-latex] (xlabel.east) -- ++(0.5,0);
\draw [-latex] (ylabel.east) -- ++(0,0.5);
\end{tikzpicture}
\end{document}
